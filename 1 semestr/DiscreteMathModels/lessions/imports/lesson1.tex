\section{Логика}

Мат.логика имеет дело с высказываниями/утверждениями. Им присваивается значение либо истина, либо ложь.

Суждения (высказывания) могут быть простыми и составными.

Выделяют базовые связки?
\begin{itemize}
    \item Не $\urcorner$
    \item И $\wedge$ (конъюнкция)
    \item Или $\vee$ (дизюнкция)
    \item Если, то $\Longrightarrow$ (импликация)
    \item Тогда и только тогда $\Leftrightarrow$
    \item Или, или $\oplus$
\end{itemize}

Можно составить таблицы, отражающую истинность соответствующих связок.

\begin{table}[h]
    \begin{subtable}[t]{0.2\textwidth}
        \begin{tabular}[h]{c c}
            A & $\urcorner$ A \\
            1 & 0 \\
            0 & 1 \\
        \end{tabular}
        \subcaption{Таблица связки "не"}
    \end{subtable}
    \begin{subtable}[t]{0.2\textwidth}
        \begin{tabular}[t]{c c c}
            A & B & A$\wedge$B \\
            1 & 1 & 1 \\
            1 & 0 & 0\\
            0 & 1 & 0 \\
            0 & 0 & 1
        \end{tabular}
        \subcaption{Таблица "И"}
    \end{subtable}
    \begin{subtable}[t]{0.2\textwidth}
        \begin{tabular}[t]{c c c}
            A & B & A$\vee$ B \\
            1 & 0 & 1 \\
            1 & 1 & 1 \\
            0 & 1 & 1 \\
            0 & 0 & 0
        \end{tabular}
        \subcaption{Таблица "или"}
    \end{subtable}
\end{table}

\begin{table}[h]
    \begin{subtable}[t]{0.2\textwidth}
        \begin{tabular}[h]{c c c}
            A & B & A$\Longrightarrow$ B \\
            1 & 1 & 1 \\
            1 & 0 & 0 \\
            0 & 1 & 1 \\
            0 & 0 & 1
        \end{tabular}
    \subcaption{Таблица "если, то"}
    \end{subtable}
    \begin{subtable}[t]{0.2\textwidth}
    \begin{tabular}[h]{c c c}
        A & B & A$\Leftrightarrow$ B \\
        1 & 0 & 0 \\
        1 & 1 & 1 \\
        0 & 1 & 0 \\
        0 & 0 & 1
    \end{tabular}
    \subcaption{Таблица "тогда и только тогда"}
    \end{subtable}
    \begin{subtable}[t]{0.2\textwidth}
    \begin{tabular}[h]{c c c}
        A & B & A$\oplus$ B \\
        1 & 0 & 1 \\
        1 & 1 & 0 \\
        0 & 1 & 1 \\
        0 & 0 & 0
    \end{tabular}
    \subcaption{"Или, или"}
    \end{subtable}
\end{table}

\subsection{Множества}

$x\in A$ - $x$ является элементом множества $A$

$x\notin A$ - не является

$B\subset A$ - множество B входит в множества А. Обозначение $\subseteq$ подразумевает возможное совпадение множества.

Например, : $A := \{1,2,3\};\; B:=\{2,3\};\;\Rightarrow B\subseteq A$

Пустое множество: $\emptyset$ или $\oslash$ или $\varnothing$ - множество, не содержащее элементов.

Можно рассматривать дополнение к множеству. Например, если работаем над полем целых чисел $\mathbb{Z}$, и вводим множество $A:=\{2,5\}$, то дополнением $\overline{A}:=\mathbb{Z}/\{2,5\}$, где $/$ - операция вычитания.

Конъюнкция множеств является множество, состоящее из общих элементов: $A\wedge B \equiv A\cap B := \{x| x\in A \wedge x\in B\}$.

Дизюнкция определяется как объединение всех элементов обоих множеств, взятых единожды: $A\vee B \equiv A\cup B := \{x,y| x\in A \vee y\in B\}$.

Разность множеств - множество, элементы которого содержатся в первом множестве, но не содержатся во мтором: $A/B := \{x: x\in A \wedge x\notin B\}$.

Пустое множество явлется нейтральным по отношению к сложению множеством.

Симметрическая разность: $A\triangle B \equiv( A/B) \cup (B/A):= \{x| (x\in A, x\notin B )\vee (x\notin A, x\in B)\}$.

Равенство множеств определяется поэлементно: $A=B:= (A\subseteq B) \wedge (B\subseteq A)$